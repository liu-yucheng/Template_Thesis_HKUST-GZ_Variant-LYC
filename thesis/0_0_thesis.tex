% Copyright (C) 2024 Yucheng Liu. Under the AGPL 3.0 License.
% AGPL 3.0 License: https://www.gnu.org/licenses/agpl-3.0.txt .

% HKUST-GZ thesis template. Using ustthesis.cls.

% Import package set 1.
\documentclass[a4paper]{ustthesis}
\usepackage[square,numbers]{natbib}
\usepackage{xeCJK}
% \usepackage[T1]{fontenc}
% \usepackage{algorithm}
% \usepackage{algorithmic}
\usepackage{longtable}
\usepackage{url}
\usepackage{multirow}
\usepackage{graphics}
\usepackage{graphicx}
\usepackage{amsmath}
\usepackage{cases}
\usepackage{colortbl}
\usepackage{xcolor}
\usepackage{times}
\usepackage{array}
\usepackage{hyperref}

\hypersetup{
    colorlinks = false,
    pdfborderstyle = {/S/U/W 1}
}

\usepackage{pdflscape}
\DeclareMathOperator*{\argmax}{argmax}
\usepackage{fontspec}

% Font settings preset 1.
\iffalse

\setmainfont{times} [
    Path = ../fonts/Times-New-Roman/ ,
    UprightFont = *.ttf ,
    BoldFont = *bd.ttf ,
    ItalicFont = *i.ttf ,
    BoldItalicFont = *bi.ttf
]

\setsansfont{arial} [
    Path = ../fonts/Arial/ ,
    UprightFont = *.ttf ,
    BoldFont = *bd.ttf ,
    ItalicFont = *i.ttf ,
    BoldItalicFont = *bi.ttf
]

\setmonofont{cour} [
    Path = ../fonts/Courier-New/ ,
    UprightFont = *.ttf ,
    BoldFont = *bd.ttf ,
    ItalicFont = *i.ttf ,
    BoldItalicFont = *bi.ttf
]

% \setCJKmainfont{AR PL UMing HK}
\setCJKmainfont{../fonts/GoNotoCurrentSerif/GoNotoCurrentSerif.ttf}

\fi

% Font settings preset 2.
\iftrue

\setmainfont{cambria} [
    Path = ../fonts/Cambria/ ,
    UprightFont = *.ttc ,
    BoldFont = *b.ttf ,
    ItalicFont = *i.ttf ,
    BoldItalicFont = *z.ttf
]

\setsansfont{calibri} [
    Path = ../fonts/Calibri/ ,
    UprightFont = *.ttf ,
    BoldFont = *b.ttf ,
    ItalicFont = *i.ttf ,
    BoldItalicFont = *z.ttf
]

\setmonofont{cour} [
    Path = ../fonts/Courier-New/ ,
    UprightFont = *.ttf ,
    BoldFont = *bd.ttf ,
    ItalicFont = *i.ttf ,
    BoldItalicFont = *bi.ttf
]

\setCJKmainfont{../fonts/GoNotoCurrentSerif/GoNotoCurrentSerif.ttf}

\fi

% Font settings preset 3.
\iffalse

\setmainfont{GoNotoCurrentSerif} [
    Path = ../fonts/GoNotoCurrentSerif/ ,
    UprightFont = *.ttf
]

\setsansfont{GoNotoCurrent} [
    Path = ../fonts/GoNotoCurrent/ ,
    UprightFont = *-Regular.ttf ,
    BoldFont = *-Bold.ttf
]

\setmonofont{CascadiaMono} [
    Path = ../fonts/Cascadia/ ,
    UprightFont = *.ttf
]

\setCJKmainfont{../fonts/GoNotoCurrentSerif/GoNotoCurrentSerif.ttf}

\fi

% Import package set 2.
% \usepackage{setspace}
% \usepackage{hyperref}
\usepackage{diagbox}
% \usepackage[T1]{fontenc}  % use 8-bit T1 fonts
% \usepackage{hyperref}     % hyperlinks
% \usepackage{url}          % simple URL typesetting
\usepackage{booktabs}       % professional-quality tables
\usepackage{amsfonts}       % blackboard math symbols
\usepackage{nicefrac}       % compact symbols for 1/2, etc.
\usepackage{microtype}      % microtypographyhttps://www.overleaf.com/project/62064cfa2a4921563b535a86
\usepackage{xcolor}         % colors
\usepackage{times}
\usepackage{epsfig}
\usepackage{graphicx}
\usepackage{amsmath}
\usepackage{amssymb}

% Import package set 3.
\usepackage{amsmath}
\usepackage{amsfonts}
\usepackage{dsfont}
\usepackage{multirow}
\usepackage{adjustbox}
\usepackage{wrapfig}
\usepackage{threeparttable}     % Package importation needed.
% \usepackage{appendix}
\usepackage[utf8x]{inputenc}

% Define commands set 1.
% Import package set 4.
\newcommand{\sign}{\text{sign}}
\newtheorem{myDef}{Definition} 
\newtheorem{myAss}{Assumption} 
% Assumption
\newtheorem{myTheo}{Theorem}
\newtheorem{myexam}{Example}
\newtheorem{proposition}{Proposition}
\newtheorem{proof}{Proof}
\usepackage[ruled,linesnumbered]{algorithm2e}
\usepackage{subfigure}
\makeatletter
\newcommand{\rmnum}[1]{\romannumeral #1}
\newcommand{\Rmnum}[1]{\expandafter\@slowromancap\romannumeral #1@}

% Define commands set 2.
\newcommand{\etal}{\emph{et~al.}\xspace}
\newcommand{\eg}{\emph{e.g.},\xspace}
\newcommand{\ie}{\emph{i.e.},\xspace}
\newcommand{\wrt}{\emph{w.r.t.}\xspace}
\newcommand{\aka}{\emph{a.k.a.}\xspace}
\newcommand{\etc}{\emph{etc.}\xspace}
\newcommand{\cf}{cf.\/~}
\newcommand\figref[1]{Figure~\ref{#1}}
\newcommand\algoref[1]{Algorithm~\ref{#1}}
\newcommand\tabref[1]{Table~\ref{#1}}
\newcommand\secref[1]{Section~\ref{#1}}
\newcommand\equref[1]{Eq.~(\ref{#1})}
\DeclareMathAlphabet\mathbfcal{OMS}{cmsy}{b}{n}

% Define commands set 3.
\newtheorem{pro}{Problem}
\newcommand{\eat}[1]{}
\ifodd 1
\newcommand{\rev}[1]{{\color{purple}{#1}}}                      % Revision of the text.
\newcommand{\response}[1]{{\color{black}{#1}}}
\newcommand{\TODO}[1]{{\color{red}TODO:{#1}}}
\newcommand\beftext[1]{{\color[rgb]{0.5,0.5,0.5}{BEFORE:#1}}}
\else
\newcommand{\rev}[1]{#1}
\newcommand{\hao}[1]{#1}
\newcommand{\TODO}[1]{}
\newcommand{\beftext}[1]{#1}
\fi

% Import package set 5.
% Use the "latexsym" package when encountering the following error:
%   ! LaTeX Error: Command \??? not provided in base LaTeX2e.
% \usepackage{latexsym}
% Use the "epsf" package for including EPS files.
% \usepackage{epsf}

% Import package set 6.
% Create glossaries.
% \usepackage{glossaries}
\usepackage[style=indexgroup]{glossaries-extra}
% Copyright (C) 2024-2025 Yucheng Liu. Under the AGPL 3.0 License.
% AGPL 3.0 License: https://www.gnu.org/licenses/agpl-3.0.txt .

\makeglossaries

\newglossaryentry{hkust_gz}{
    name={HKUST(GZ)},
    description={Hong Kong University of Science and Technology (Guangzhou).}
}


% Preambles. DO NOT ERASE THEM.
\title{Title}                           % Title of the thesis.
\author{Your Name}                      % Author of the thesis.
\degree{\MPhil}                         % Degree for which the thesis is. Options: \AM \MSc \MPhil \PhD .
\stage{\Thesis}                         % Stage of PhD document. Options: \PQE \Proposal \Thesis .
\subject{Your Thrust}                   % Subject of the Degree.
\department{Your Thrust}                % Department to which the thesis is submitted.
\advisor{Prof. A}                       % Supervisor.
% co-supervisor can be added using \member .
\member{Prof. B}
%\acting                                % Uncomment for Accting department head.
\depthead{Prof. C}                      % Department head.
\defencedate{2023}{12}{18}              % \defencedate{year}{month}{day}.

% According to the sample shown in the guidelines, page number is placed below the bottom margin.
% However, if the author prefers the page number to be printed above the bottom margin,
%   please activate the following command.
%\PNumberAboveBottomMargin

\begin{document}
%\begin{CJK}{UTF8}{song}    % Bitstream Cyber Bit song ti
%\begin{CJK*}{UTF8}{gbsn}   % Arphic song ti

% Thesis Content. The order of output MUST be followed.
% 0.0. TITLEPAGE.
% The \maketitle command generates the Title page as well as the Signature page.
\maketitle

% 0.1. DEDICATION (Optional).
% The \dedication and \enddedication commands are optional.
% If specified it generates a page for dedication.
\dedication
% This is an optional section.
% Copyright (C) 2024-2025 Yucheng Liu. Under the AGPL 3.0 License.
% AGPL 3.0 License: https://www.gnu.org/licenses/agpl-3.0.txt .

\noindent You raise me up, so I can stand on mountains;\\
you raise me up, to walk on stormy seas.\\
I am strong when I am on your shoulders.\\
You raise me up, to more than I can be.\\
--- ``You raise me up'' lyrics by Brendan Graham.\\

\enddedication
\newpage

% 0.2. ACKNOWLEDGMENTS.
% \acknowledgments and \endacknowledgments defines the Acknowledgments of the author of the Thesis.
\acknowledgments
% Copyright (C) 2024-2025 Yucheng Liu. Under the AGPL 3.0 License.
% AGPL 3.0 License: https://www.gnu.org/licenses/agpl-3.0.txt .

Thanks all.

This thesis follows the ``Hong Kong University of Science and Technology (Guangzhou) - Handbook for Research Postgraduate Studies - Guidelines on Thesis Preparation\citep{ust_thesis_guidelines, ust_thesis_mphil}.''

\endacknowledgments
\newpage

% 0.3. TABLE OF CONTENTS.
\tableofcontents

% 0.4. LIST OF FIGURES (If Any).
\listoffigures

% 0.5. LIST OF TABLES (If Any).
\listoftables

% 0.6. ABSTRACT
% \abstract and \endabstract are used to define a short Abstract for the Thesis.
\abstract
\input{0_6_abstract.tex}
\endabstract

% 1. The Actual Contents
% The command \chapters MUST BE USED to ensure that the entire content of the Thesis is double-spaced
%   (in version 1.0).
% However, in version 2.0, \chapters will be automatically added in the beginning of the first chapter.
% \chapters     % Not necessary with ustthesis.cls (v2.0).
% Each chapter is defined via the \chapter command.
% The usual sectional commands of LaTeX are also available.

% 1.1. Introduction.
\chapter{Introduction}
\label{chapter_intro}
% Copyright (C) 2024-2025 Yucheng Liu. Under the AGPL 3.0 License.
% AGPL 3.0 License: https://www.gnu.org/licenses/agpl-3.0.txt .

Introduction.

\section{Inline Code Listing Tests}
\label{sec_InlineCodeListingTests}

Here are some inline code texts.
\lstinline|Lorem ipsum dolor sit amet, consectetur adipiscing elit, sed do eiusmod tempor incididunt ut labore et dolore magna aliqua. Ut enim ad minim veniam, quis nostrud exercitation ullamco laboris nisi ut aliquip ex ea commodo consequat. Duis aute irure dolor in reprehenderit in voluptate velit esse cillum dolore eu fugiat nulla pariatur. Excepteur sint occaecat cupidatat non proident, sunt in culpa qui officia deserunt mollit anim id est laborum.|
This is the end of the inline code texts.

\newpage

% 1.2. Guidelines (DO NOT include).
% \chapter{Guidelines}
% \label{chapter_guidelines}
% \input{1_2_guidelines.tex}
% \newpage

% 1.3. Literature Review.
\chapter{Literature Review}
\label{chapter_review}
% Copyright (C) 2024 Yucheng Liu. Under the AGPL 3.0 License.
% AGPL 3.0 License: https://www.gnu.org/licenses/agpl-3.0.txt .

Review.

\newpage

% 1.4. Methodologies.
\chapter{Methodologies}
\label{chapter_methodologies}
\input{1_4_methodologies.tex}
\newpage

% 1.5. Experiments.
\chapter{Experiments}
\label{chapter_experiments}
\input{1_5_experiments.tex}
\newpage

% 1.6. Discussions.
\chapter{Discussions}
\label{chapter_discussions}
\input{1_6_discussions.tex}
\newpage

% 1.7. Limitations.
\chapter{Limitations}
\label{chapter_limitations}
\input{1_7_limitations.tex}
\newpage

% 1.8. Conclusions.
\chapter{Conclusions}
\label{chapter_conclusions}
\input{1_8_conclusions.tex}
\newpage

% 2.1. REFERENCES.
%
% This example uses bibtex to generate the required Bibliography.
% Refer to the file ustthesis_test.bib for the entries of the Bibliography.
% Note that only the cited entries are printed.
%                                 %
% If BibTeX is not used to typeset the bibliography,
%   replace the following line with the \begin{thebibliography} and \end{bibliography}% commands
%   (the "thebibliography" environment) to process the Bibliography.
%
% The recommended bibliography style is the IEEE bibliography style.
% "ustbib" defines the IEEE bibliography standard with the added ability of sorting the items by name of author.
%
% If you are not using BibTeX to process your Bibliography, comment out the following line.
\addcontentsline{toc}{chapter}{Bibliography}
\bibliographystyle{IEEEtran}
\bibliography{2_1_references.bib}
\newpage

% 2.2. APPENDICES (If Any)
% \appendix command marks the beginning of the APPENDIX part of the Thesis.
% The usual \chapter command is used for the different chapters of the Appendix.
\appendix
\chapter{Additional Materials}
\label{chapter_additional_materials}
\input{2_2_additional_materials.tex}
\newpage

% 2.3. BIOGRAPHY (optional)
% \biography and \endbiography are used to define the optional Biography of the author of the Thesis.
% \label{chapter_biography}
% \biography
% \endbiography
% \newpage

% 2.4. GLOSSARIES (optional)
\chapter{Glossaries}
\label{chapter_glossaries}
\input{2_4_2_glossaries_texts.tex}
% \newpage

\end{document}
